\section{REST in Rails}

\imageFrame{images/rest-in-rails.jpg}
\centeredFrame{REST in Rails}
\centeredFrame{in Rails it's easier to build RESTful\\than non-RESTful apps}

\centeredFrame{quick demo}
% create Rails app, scaffold posts, turn off forgery protection
% curl -X POST -H 'Accept: application/xml' -d post[title]='test' -d post[body]='test' http://localhost:3000/posts
% curl -X GET -H 'Accept: application/xml' http://localhost:3000/posts/x
% curl -X PUT -H 'Accept: application/xml' -d post[title]='updated with curl' http://localhost:3000/posts/x
% curl -X GET -H 'Accept: application/xml' http://localhost:3000/posts/x
% curl -X DELETE -H 'Accept: application/xml' http://localhost:3000/posts/x

\centeredFrame{how does it work?}

\begin{frame}
  \frametitle{\insertsection}
  \begin{columns}
    \begin{column}{3cm}
      \begin{center}REST actions\\\end{center}
        POST\\
        GET\\
        PUT\\
        DELETE
    \end{column}
  \end{columns}
\end{frame}

\begin{frame}
  \frametitle{\insertsection}
  \begin{columns}
    \begin{column}{3cm}
      \begin{center}Rails actions\\\end{center}
        create\\
        show\\
        update\\
        destroy\\
        \uncover<2->{new\\}
        \uncover<2->{edit\\}
        \uncover<2->{index\\}
    \end{column}
  \end{columns}
\end{frame}

\centeredFrame{7 default actions}

\begin{frame}
  \frametitle{\insertsection}
  \begin{columns}
    \begin{column}{2.5cm}
      \begin{center}Rails actions\\\end{center}
        create\\
        show\\
        update\\
        destroy\\
        index\\
        new\\
        edit
    \end{column}
    \begin{column}{3.5cm}
      \begin{center}HTTP request\\\end{center}
        \begin{columns}
          \begin{column}{1.5cm}
            POST\\
            GET\\
            PUT\\
            DELETE\\
            GET\\
            GET\\
            GET
          \end{column}
          \begin{column}{2cm}
            /users\\
            /users/1\\
            /users/1\\
            /users/1\\
            /users\\
            /users/1/new\\
            /users/1/edit
          \end{column}
        \end{columns}
    \end{column}
  \end{columns}
\end{frame}

\centeredFrame{how does Rails know how to map URI to an action?}
\centeredFrame{routes}

\inputFrame{config/routes.rb}{}{files/routes}

\centeredFrame{
  generates mapping for 7 default actions\\
  for user resource
}

\centeredFrame{
  generates helper methods for 7 default actions\\
  for user resource
}

\begin{frame}
  \frametitle{\insertsection}
  \begin{columns}
    \begin{column}{2cm}
      \begin{center}Rails actions\end{center}
        create\\
        show\\
        update\\
        destroy\\
        index\\
        new\\
        edit
    \end{column}
    \begin{column}{2cm}
      \begin{center}URI\end{center}
      /users\\
      /users/1\\
      /users/1\\
      /users/1\\
      /users\\
      /users/1/new\\
      /users/1/edit
    \end{column}
    \begin{column}{2cm}
      \begin{center}helpers\end{center}
        users\_path\\
        user\_path(1)\\
        user\_path(1)\\
        user\_path(1)\\
        users\_path\\
        new\_user\_path\\
        edit\_user\_path(1)
    \end{column}
  \end{columns}
\end{frame}

\centeredFrame{resource representations}
\centeredFrame{respond\_to}

\inputFrame{app/controllers/users\_controller.rb}{\small}{files/respond-to}
